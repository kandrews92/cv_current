% resume.tex
% vim:set ft=tex spell:

\documentclass[10pt,letterpaper]{article}
\usepackage[letterpaper,margin=0.75in]{geometry}
\usepackage[utf8]{inputenc}
\usepackage{mdwlist}
\usepackage[T1]{fontenc}
\usepackage{textcomp}
\usepackage{chemformula}
\usepackage{tgpagella}
\pagestyle{empty}
\setlength{\tabcolsep}{0em}

% indentsection style, used for sections that aren't already in lists
% that need indentation to the level of all text in the document
\newenvironment{indentsection}[1]%
{\begin{list}{}%
	{\setlength{\leftmargin}{#1}}%
	\item[]%
}
{\end{list}}

% opposite of above; bump a section back toward the left margin
\newenvironment{unindentsection}[1]%
{\begin{list}{}%
	{\setlength{\leftmargin}{-0.5#1}}%
	\item[]%
}
{\end{list}}

% format two pieces of text, one left aligned and one right aligned
\newcommand{\headerrow}[2]
{\begin{tabular*}{\linewidth}{l@{\extracolsep{\fill}}r}
	#1 &
	#2 \\
\end{tabular*}}

% make "C++" look pretty when used in text by touching up the plus signs
\newcommand{\CPP}
{C\nolinebreak[4]\hspace{-.05em}\raisebox{.22ex}{\footnotesize\bf ++}}

% and the actual content starts here
\begin{document}

%%%
%%% contact info and headings
%%%
\begin{center}
{\LARGE \textbf{Kraig J. Andrews}}

61943 Mustang Drive\ \ \textbullet
%\ \ Suite\ 1001\ \ \textbullet
\ \ South Lyon, MI 48178
\\
(248) 798-9388\ \ \textbullet
\ \ kraigandrews1992@gmail.com
\end{center}

%%%
%%% relevant education
%%%
\hrule
\vspace{-0.4em}
\subsection*{Education}

\begin{itemize}
	\parskip=0.1em

    % Phd
    \item 
    \headerrow
        {\textbf{Wayne State University}}
        {\textbf{Detroit, MI}}
    \\
    \headerrow
        {\emph{Department of Physics \& Astronomy, Ph.D. Physics}}
        {\emph{2014 -- Present}}
    \begin{itemize*}
        \item\emph{Advisor:} Dr. Zhixian Zhou
        \item\emph{Thesis Title:} ``Quantum Transport Properties and Scattering Mechanisms in Transition Metal Dichalcogenides'' 
    \end{itemize*}
    
    % Masters
    \item 
    \headerrow
        {\textbf{Wayne State University}}
        {\textbf{Detroit, MI}}
    \\
    \headerrow
        {\emph{Department of Physics \& Astronomy, M.S. Physics}}
        {\emph{2017}}

    % B.S. number 1
	\item 
	\headerrow
		{\textbf{Michigan State University}}
		{\textbf{East Lansing, MI}}
	\\
	\headerrow
		{\emph{Department of Physics \& Astronomy, B.S. Physics}}
        {\emph{2014}}
    
    % B.S. number 2
    \item 
    \headerrow
        {\textbf{Michigan State University}}
        {\textbf{East Lansing, MI}}
    \\
    \headerrow
        {\emph{Department of Physics \& Astronomy, B.S. Astrophysics}}
        {\emph{2014}}

\end{itemize}

%%%
%%% relevant experience
%%%
\hrule
\vspace{-0.4em}
\subsection*{Experience}

\begin{itemize}
	\parskip=0.1em

	\item
	\headerrow
		{\textbf{Nano Fabrication \& Electron Transport Laboratory}}
		{\textbf{Wayne State University, Detroit, MI}}
	\\
	\headerrow
		{\emph{Graduate Research Assistant}}
		{\emph{2015 -- Present}}
	\begin{itemize*}
		\item Fabricate two-dimensional field effect transistors using transition metal dichalcogenides, such as 
		molybdenum disulphide, tungsten diselenide, and molybdenum diselenide to investigate intrinsic transport properties.
		\item Develop novel techniques for making low-resistance Ohmic contacts to a wide variety of 
		two-dimensional semiconductors.
	\end{itemize*}

	\item
	\headerrow
		{\textbf{National Institute of Materials Science}}
		{\textbf{Tsukuba, Ibaraki, Japan}}
	\\
	\headerrow
		{\emph{Visiting Graduate Researcher, Summer Intern}}
		{\emph{2017}}
	\begin{itemize*}
		\item Investigate methods for surface modification of two-dimensional semiconductors for
		the use of creating a new low-resistance Ohmic contact strategy.
	\end{itemize*}

	\item
	\headerrow
		{\textbf{Interational Course on Computational Physics}}
		{\textbf{Delft, Netherlands \& East Lansing, MI}}
	\\
	\headerrow
		{\emph{Undergraduate Researcher}}
		{\emph{2014}}
	\begin{itemize*}
		\item A Joint collaboration with Technische Universiteit Delft and Michigan State University involving the development of 
		computational models of various physical systems to model interactions of materials and optimize employed techniques.
	\end{itemize*}

\end{itemize}

%%%
%%% relevant talks
%%%
%\hrule
%\vspace{-0.4em}
%\subsection*{Selected Publications}

%\begin{enumerate*}
%	\parskip=0.1em
%	\item
%\end{enumerate*}

%%%
%%% relevant talks
%%%
%\hrule
%\vspace{-0.4em}
%\subsection*{Selected Presentations}

%\begin{enumerate*}
%	\parskip=0.1em
%	\item
%\end{enumerate*}

%%%
%%% courses taught
%%%
%\hrule
%\vspace{-0.4em}
%\subsection*{Teaching Experience}

%\begin{itemize*}
%	\parskip=0.1em
%	\item
%\end{itemize*}

%%%
%%% relevant skills
%%%
\hrule
\vspace{-0.4em}
\subsection*{Core Technical Skills}

\begin{indentsection}{\parindent}
\hyphenpenalty=1000
\begin{description*}
	\item[Nanofabrication:]
	Atomic force microscopy (AFM), Electron beam lithography, Photolithography, Scanning electron microscopy (SEM), General clean room abilities, Physical vapor deposition (PVD), Electron beam deposition, Plasma etching, Reactive ion etching (RIE)
	\item[Languages \& Software:]
	\CPP, Fortran, Java, JavaScript, \LaTeX, Python, shell script, Microsoft Office, Matlab, Mathematica
	\item[Operating Systems:]
	OS X, Linux OS, Microsoft Windows
\end{description*}
\end{indentsection}

\end{document}
